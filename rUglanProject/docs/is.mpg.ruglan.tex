\documentclass[11pt,a4paper]{report}
\usepackage{color}
\usepackage{ifthen}
\usepackage{ifpdf}
\usepackage[headings]{fullpage}
\ifpdf \usepackage[pdftex, pdfpagemode={UseOutlines},bookmarks,colorlinks,linkcolor={blue},plainpages=false,pdfpagelabels,citecolor={red},breaklinks=true]{hyperref}
  \usepackage[pdftex]{graphicx}
  \pdfcompresslevel=9
  \DeclareGraphicsRule{*}{mps}{*}{}
\else
  \usepackage[dvips]{graphicx}
\fi

\newcommand{\entityintro}[3]{%
  \hbox to \hsize{%
    \vbox{%
      \hbox to .2in{}%
    }%
    {\bf  #1}%
    \dotfill\pageref{#2}%
  }
  \makebox[\hsize]{%
    \parbox{.4in}{}%
    \parbox[l]{5in}{%
      \vspace{1mm}%
      #3%
      \vspace{1mm}%
    }%
  }%
}
\newcommand{\refdefined}[1]{
\expandafter\ifx\csname r@#1\endcsname\relax
\relax\else
{$($in \ref{#1}, page \pageref{#1}$)$}\fi}
\date{\today}
\title{rUglan documentation}
\author{Jon Arnar Tomasson, Matthias Pall Gissurarson and Sigurdur Fannar Vilhelmsson}
\chardef\textbackslash=`\\
\begin{document}
\maketitle
\sloppy
\addtocontents{toc}{\protect\markboth{Contents}{Contents}}
\tableofcontents
\chapter*{Class Hierarchy}{
\thispagestyle{empty}
\markboth{Class Hierarchy}{Class Hierarchy}
\addcontentsline{toc}{chapter}{Class Hierarchy}
\subsection*{Classes}
{\raggedright
\hspace{0.0cm} $\bullet$ java.lang.Object {\tiny \refdefined{java.lang.Object}} \\
\hspace{1.0cm} $\bullet$  {\tiny } \\
\hspace{2.0cm} $\bullet$ is.mpg.ruglan.iCalParser {\tiny \refdefined{is.mpg.ruglan.iCalParser}} \\
\hspace{1.0cm} $\bullet$ Activity {\tiny } \\
\hspace{2.0cm} $\bullet$ is.mpg.ruglan.CalEventActivity {\tiny \refdefined{is.mpg.ruglan.CalEventActivity}} \\
\hspace{2.0cm} $\bullet$ is.mpg.ruglan.HomeActivity {\tiny \refdefined{is.mpg.ruglan.HomeActivity}} \\
\hspace{2.0cm} $\bullet$ is.mpg.ruglan.SettingsActivity {\tiny \refdefined{is.mpg.ruglan.SettingsActivity}} \\
\hspace{1.0cm} $\bullet$ SQLiteOpenHelper {\tiny } \\
\hspace{2.0cm} $\bullet$ is.mpg.ruglan.rDataBase {\tiny \refdefined{is.mpg.ruglan.rDataBase}} \\
\hspace{1.0cm} $\bullet$ is.mpg.ruglan.CalEvent {\tiny \refdefined{is.mpg.ruglan.CalEvent}} \\
\hspace{1.0cm} $\bullet$ is.mpg.ruglan.Dabbi {\tiny \refdefined{is.mpg.ruglan.Dabbi}} \\
}
}
\chapter{Package is.mpg.ruglan}{
\label{is.mpg.ruglan}\hskip -.05in
\hbox to \hsize{\textit{ Package Contents\hfil Page}}
\vskip .13in
\hbox{{\bf  Classes}}
\entityintro{CalEvent}{is.mpg.ruglan.CalEvent}{Class representing Calendar Event for the project.}
\entityintro{CalEventActivity}{is.mpg.ruglan.CalEventActivity}{}
\entityintro{Dabbi}{is.mpg.ruglan.Dabbi}{An interface for the database backend.}
\entityintro{HomeActivity}{is.mpg.ruglan.HomeActivity}{}
\entityintro{iCalParser}{is.mpg.ruglan.iCalParser}{Parser for iCalendars.}
\entityintro{rDataBase}{is.mpg.ruglan.rDataBase}{An helper class for an sqlite datebase All credit to Gaddo F.}
\entityintro{SettingsActivity}{is.mpg.ruglan.SettingsActivity}{}
\vskip .1in
\vskip .1in
\subsection{\label{is.mpg.ruglan.CalEvent}Class CalEvent}{
\vskip .1in 
Class representing Calendar Event for the project.\vskip .1in 
\subsubsection{Declaration}{
\small public class CalEvent
\\ {\bf  extends} java.lang.Object
\refdefined{java.lang.Object}\\ {\bf  implements} 
java.io.Serializable}
\subsubsection{Constructor summary}{
\begin{verse}
{\bf CalEvent(String, String, String, Date, Date)} \\
\end{verse}
}
\subsubsection{Method summary}{
\begin{verse}
{\bf equals(CalEvent)} \\
{\bf getColor()} \\
{\bf getDescription()} \\
{\bf getDurationString()} \\
{\bf getEnd()} \\
{\bf getFullCalendarEndDateString()} \\
{\bf getFullCalendarStartDateString()} \\
{\bf getLocation()} \\
{\bf getName()} \\
{\bf getStart()} \\
{\bf toString()} \\
\end{verse}
}
\subsubsection{Constructors}{
\vskip -2em
\begin{itemize}
\item{ 
\index{CalEvent(String, String, String, Date, Date)}
{\bf  CalEvent}\\
\texttt{public\ {\bf  CalEvent}(\texttt{java.lang.String} {\bf  name},
\texttt{java.lang.String} {\bf  description},
\texttt{java.lang.String} {\bf  location},
\texttt{java.util.Date} {\bf  start},
\texttt{java.util.Date} {\bf  end})
\label{is.mpg.ruglan.CalEvent(java.lang.String, java.lang.String, java.lang.String, java.util.Date, java.util.Date)}}%end signature
\begin{itemize}
\item{
{\bf  Parameters}
  \begin{itemize}
   \item{
\texttt{name} -- Name of the event.}
   \item{
\texttt{description} -- Description of the event.}
   \item{
\texttt{location} -- Location of the event.}
   \item{
\texttt{start} -- Start date of the event.}
   \item{
\texttt{end} -- End date of the event.}
  \end{itemize}
}%end item
\end{itemize}
}%end item
\end{itemize}
}
\subsubsection{Methods}{
\vskip -2em
\begin{itemize}
\item{ 
\index{equals(CalEvent)}
{\bf  equals}\\
\texttt{public boolean\ {\bf  equals}(\texttt{CalEvent} {\bf  a})
\label{is.mpg.ruglan.CalEvent.equals(is.mpg.ruglan.CalEvent)}}%end signature
\begin{itemize}
\item{{\bf  Returns} -- 
true if a and b have the same properties, false otherwise 
}%end item
\end{itemize}
}%end item
\item{ 
\index{getColor()}
{\bf  getColor}\\
\texttt{public java.lang.String\ {\bf  getColor}()
\label{is.mpg.ruglan.CalEvent.getColor()}}%end signature
\begin{itemize}
\item{{\bf  Returns} -- 
s is a string representing a color that can be used as a parameter for backgroundColor in FullCalendar 
}%end item
\end{itemize}
}%end item
\item{ 
\index{getDescription()}
{\bf  getDescription}\\
\texttt{public java.lang.String\ {\bf  getDescription}()
\label{is.mpg.ruglan.CalEvent.getDescription()}}%end signature
\begin{itemize}
\item{{\bf  Returns} -- 
Description of the event. 
}%end item
\end{itemize}
}%end item
\item{ 
\index{getDurationString()}
{\bf  getDurationString}\\
\texttt{public java.lang.String\ {\bf  getDurationString}()
\label{is.mpg.ruglan.CalEvent.getDurationString()}}%end signature
\begin{itemize}
\item{{\bf  Returns} -- 
A string on the form "HH:MM - HH:MM" describing the duration of the event. 
}%end item
\end{itemize}
}%end item
\item{ 
\index{getEnd()}
{\bf  getEnd}\\
\texttt{public java.util.Date\ {\bf  getEnd}()
\label{is.mpg.ruglan.CalEvent.getEnd()}}%end signature
\begin{itemize}
\item{{\bf  Returns} -- 
End date of the event. 
}%end item
\end{itemize}
}%end item
\item{ 
\index{getFullCalendarEndDateString()}
{\bf  getFullCalendarEndDateString}\\
\texttt{public java.lang.String\ {\bf  getFullCalendarEndDateString}()
\label{is.mpg.ruglan.CalEvent.getFullCalendarEndDateString()}}%end signature
\begin{itemize}
\item{{\bf  Returns} -- 
s is a string on the form "new Date(y, m, d, H, M)" that can be used as a parameter for date in FullCalendar 
}%end item
\end{itemize}
}%end item
\item{ 
\index{getFullCalendarStartDateString()}
{\bf  getFullCalendarStartDateString}\\
\texttt{public java.lang.String\ {\bf  getFullCalendarStartDateString}()
\label{is.mpg.ruglan.CalEvent.getFullCalendarStartDateString()}}%end signature
\begin{itemize}
\item{{\bf  Returns} -- 
s is a string on the form "new Date(y, m, d, H, M)" that can be used as a parameter for date in FullCalendar 
}%end item
\end{itemize}
}%end item
\item{ 
\index{getLocation()}
{\bf  getLocation}\\
\texttt{public java.lang.String\ {\bf  getLocation}()
\label{is.mpg.ruglan.CalEvent.getLocation()}}%end signature
\begin{itemize}
\item{{\bf  Returns} -- 
Location of the event. 
}%end item
\end{itemize}
}%end item
\item{ 
\index{getName()}
{\bf  getName}\\
\texttt{public java.lang.String\ {\bf  getName}()
\label{is.mpg.ruglan.CalEvent.getName()}}%end signature
\begin{itemize}
\item{{\bf  Returns} -- 
Name of the event. 
}%end item
\end{itemize}
}%end item
\item{ 
\index{getStart()}
{\bf  getStart}\\
\texttt{public java.util.Date\ {\bf  getStart}()
\label{is.mpg.ruglan.CalEvent.getStart()}}%end signature
\begin{itemize}
\item{{\bf  Returns} -- 
Start date of the event. 
}%end item
\end{itemize}
}%end item
\item{ 
\index{toString()}
{\bf  toString}\\
\texttt{public java.lang.String\ {\bf  toString}()
\label{is.mpg.ruglan.CalEvent.toString()}}%end signature
}%end item
\end{itemize}
}
}
\subsection{\label{is.mpg.ruglan.CalEventActivity}Class CalEventActivity}{
\vskip .1in 
\subsubsection{Declaration}{
\small public class CalEventActivity
\\ {\bf  extends} Activity
\refdefined{.Activity}}
\subsubsection{Constructor summary}{
\begin{verse}
{\bf CalEventActivity()} \\
\end{verse}
}
\subsubsection{Method summary}{
\begin{verse}
{\bf onCreate(Bundle)} \\
{\bf onCreateOptionsMenu(Menu)} \\
{\bf onOptionsItemSelected(MenuItem)} \\
\end{verse}
}
\subsubsection{Constructors}{
\vskip -2em
\begin{itemize}
\item{ 
\index{CalEventActivity()}
{\bf  CalEventActivity}\\
\texttt{public\ {\bf  CalEventActivity}()
\label{is.mpg.ruglan.CalEventActivity()}}%end signature
}%end item
\end{itemize}
}
\subsubsection{Methods}{
\vskip -2em
\begin{itemize}
\item{ 
\index{onCreate(Bundle)}
{\bf  onCreate}\\
\texttt{protected void\ {\bf  onCreate}(\texttt{Bundle} {\bf  savedInstanceState})
\label{is.mpg.ruglan.CalEventActivity.onCreate(Bundle)}}%end signature
}%end item
\item{ 
\index{onCreateOptionsMenu(Menu)}
{\bf  onCreateOptionsMenu}\\
\texttt{public boolean\ {\bf  onCreateOptionsMenu}(\texttt{Menu} {\bf  menu})
\label{is.mpg.ruglan.CalEventActivity.onCreateOptionsMenu(Menu)}}%end signature
}%end item
\item{ 
\index{onOptionsItemSelected(MenuItem)}
{\bf  onOptionsItemSelected}\\
\texttt{public boolean\ {\bf  onOptionsItemSelected}(\texttt{MenuItem} {\bf  item})
\label{is.mpg.ruglan.CalEventActivity.onOptionsItemSelected(MenuItem)}}%end signature
}%end item
\end{itemize}
}
}
\subsection{\label{is.mpg.ruglan.Dabbi}Class Dabbi}{
\vskip .1in 
An interface for the database backend.\vskip .1in 
\subsubsection{Declaration}{
\small public class Dabbi
\\ {\bf  extends} java.lang.Object
\refdefined{java.lang.Object}}
\subsubsection{Constructor summary}{
\begin{verse}
{\bf Dabbi()} \\
{\bf Dabbi(Context)} \\
\end{verse}
}
\subsubsection{Method summary}{
\begin{verse}
{\bf addCalEvents(CalEvent\lbrack \rbrack )} \\
{\bf getCalEvents(Date, Date)} \\
\end{verse}
}
\subsubsection{Constructors}{
\vskip -2em
\begin{itemize}
\item{ 
\index{Dabbi()}
{\bf  Dabbi}\\
\texttt{public\ {\bf  Dabbi}()
\label{is.mpg.ruglan.Dabbi()}}%end signature
}%end item
\item{ 
\index{Dabbi(Context)}
{\bf  Dabbi}\\
\texttt{public\ {\bf  Dabbi}(\texttt{Context} {\bf  context})
\label{is.mpg.ruglan.Dabbi(Context)}}%end signature
\begin{itemize}
\item{
{\bf  Parameters}
  \begin{itemize}
   \item{
\texttt{context} -- A non null android Context object.}
  \end{itemize}
}%end item
\end{itemize}
}%end item
\end{itemize}
}
\subsubsection{Methods}{
\vskip -2em
\begin{itemize}
\item{ 
\index{addCalEvents(CalEvent\lbrack \rbrack )}
{\bf  addCalEvents}\\
\texttt{public void\ {\bf  addCalEvents}(\texttt{CalEvent\lbrack \rbrack } {\bf  calEvents})
\label{is.mpg.ruglan.Dabbi.addCalEvents(is.mpg.ruglan.CalEvent[])}}%end signature
\begin{itemize}
\item{
{\bf  Parameters}
  \begin{itemize}
   \item{
\texttt{calEvents} -- An array of CalEvents to be added.}
  \end{itemize}
}%end item
\end{itemize}
}%end item
\item{ 
\index{getCalEvents(Date, Date)}
{\bf  getCalEvents}\\
\texttt{public CalEvent\lbrack \rbrack \ {\bf  getCalEvents}(\texttt{java.util.Date} {\bf  start},
\texttt{java.util.Date} {\bf  end})
\label{is.mpg.ruglan.Dabbi.getCalEvents(java.util.Date, java.util.Date)}}%end signature
\begin{itemize}
\item{
{\bf  Parameters}
  \begin{itemize}
   \item{
\texttt{start} -- A Date object that is the earliest date we want to look at.}
   \item{
\texttt{end} -- A Date object that is the latest time an CalEvent can start at so it is included in the return value.}
  \end{itemize}
}%end item
\end{itemize}
}%end item
\end{itemize}
}
}
\subsection{\label{is.mpg.ruglan.HomeActivity}Class HomeActivity}{
\vskip .1in 
\subsubsection{Declaration}{
\small public class HomeActivity
\\ {\bf  extends} Activity
\refdefined{.Activity}}
\subsubsection{Field summary}{
\begin{verse}
{\bf events} \\
\end{verse}
}
\subsubsection{Constructor summary}{
\begin{verse}
{\bf HomeActivity()} \\
\end{verse}
}
\subsubsection{Method summary}{
\begin{verse}
{\bf getContext()} \\
{\bf isURLMatching(String)} \\
{\bf onCreate(Bundle)} \\
{\bf onCreateOptionsMenu(Menu)} \\
{\bf openCalEventActivity(String)} \\
\end{verse}
}
\subsubsection{Fields}{
\begin{itemize}
\item{
\index{events}
\label{is.mpg.ruglan.HomeActivity.events} CalEvent {\bf  events}}
\end{itemize}
}
\subsubsection{Constructors}{
\vskip -2em
\begin{itemize}
\item{ 
\index{HomeActivity()}
{\bf  HomeActivity}\\
\texttt{public\ {\bf  HomeActivity}()
\label{is.mpg.ruglan.HomeActivity()}}%end signature
}%end item
\end{itemize}
}
\subsubsection{Methods}{
\vskip -2em
\begin{itemize}
\item{ 
\index{getContext()}
{\bf  getContext}\\
\texttt{public static Context\ {\bf  getContext}()
\label{is.mpg.ruglan.HomeActivity.getContext()}}%end signature
}%end item
\item{ 
\index{isURLMatching(String)}
{\bf  isURLMatching}\\
\texttt{protected boolean\ {\bf  isURLMatching}(\texttt{java.lang.String} {\bf  url})
\label{is.mpg.ruglan.HomeActivity.isURLMatching(java.lang.String)}}%end signature
}%end item
\item{ 
\index{onCreate(Bundle)}
{\bf  onCreate}\\
\texttt{protected void\ {\bf  onCreate}(\texttt{Bundle} {\bf  savedInstanceState})
\label{is.mpg.ruglan.HomeActivity.onCreate(Bundle)}}%end signature
}%end item
\item{ 
\index{onCreateOptionsMenu(Menu)}
{\bf  onCreateOptionsMenu}\\
\texttt{public boolean\ {\bf  onCreateOptionsMenu}(\texttt{Menu} {\bf  menu})
\label{is.mpg.ruglan.HomeActivity.onCreateOptionsMenu(Menu)}}%end signature
}%end item
\item{ 
\index{openCalEventActivity(String)}
{\bf  openCalEventActivity}\\
\texttt{protected void\ {\bf  openCalEventActivity}(\texttt{java.lang.String} {\bf  url})
\label{is.mpg.ruglan.HomeActivity.openCalEventActivity(java.lang.String)}}%end signature
}%end item
\end{itemize}
}
}
\subsection{\label{is.mpg.ruglan.iCalParser}Class iCalParser}{
\vskip .1in 
Parser for iCalendars.\vskip .1in 
\subsubsection{Declaration}{
\small public class iCalParser
\\ {\bf  extends} 
\refdefined{.<any>}}
\subsubsection{Constructor summary}{
\begin{verse}
{\bf iCalParser()} \\
\end{verse}
}
\subsubsection{Method summary}{
\begin{verse}
{\bf calendarToEventList(String)} Takes an iCal calendar string, and returns an array of string arrays each containing an event\\
{\bf doInBackground(String\lbrack \rbrack )} \\
{\bf getICalValue(String\lbrack \rbrack , String)} Takes an string array containing an iCal event, and returns the value of the type\\
{\bf parseCalendar(String)} Takes a string containing an iCal calendar and returns a list of the events in the calendar\\
{\bf removeCR(String\lbrack \rbrack )} Removes carriage returns in all string in an array of strings\\
{\bf urlToCalEvents(String)} \\
{\bf urlToString(String)} Takes an url and fetches its contents into a string\\
\end{verse}
}
\subsubsection{Constructors}{
\vskip -2em
\begin{itemize}
\item{ 
\index{iCalParser()}
{\bf  iCalParser}\\
\texttt{public\ {\bf  iCalParser}()
\label{is.mpg.ruglan.iCalParser()}}%end signature
}%end item
\end{itemize}
}
\subsubsection{Methods}{
\vskip -2em
\begin{itemize}
\item{ 
\index{calendarToEventList(String)}
{\bf  calendarToEventList}\\
\texttt{public static java.lang.String\lbrack \rbrack \lbrack \rbrack \ {\bf  calendarToEventList}(\texttt{java.lang.String} {\bf  calendar})
\label{is.mpg.ruglan.iCalParser.calendarToEventList(java.lang.String)}}%end signature
\begin{itemize}
\item{
{\bf  Description}

Takes an iCal calendar string, and returns an array of string arrays each containing an event
}
\item{
{\bf  Parameters}
  \begin{itemize}
   \item{
\texttt{calendar} -- a string that contains an iCal calendar}
  \end{itemize}
}%end item
\item{{\bf  Returns} -- 
an array of the events in the iCal calendar calendar. 
}%end item
\end{itemize}
}%end item
\item{ 
\index{doInBackground(String\lbrack \rbrack )}
{\bf  doInBackground}\\
\texttt{public CalEvent\lbrack \rbrack \ {\bf  doInBackground}(\texttt{java.lang.String\lbrack \rbrack } {\bf  url})
\label{is.mpg.ruglan.iCalParser.doInBackground(java.lang.String[])}}%end signature
}%end item
\item{ 
\index{getICalValue(String\lbrack \rbrack , String)}
{\bf  getICalValue}\\
\texttt{public static java.lang.String\ {\bf  getICalValue}(\texttt{java.lang.String\lbrack \rbrack } {\bf  event},
\texttt{java.lang.String} {\bf  type})
\label{is.mpg.ruglan.iCalParser.getICalValue(java.lang.String[], java.lang.String)}}%end signature
\begin{itemize}
\item{
{\bf  Description}

Takes an string array containing an iCal event, and returns the value of the type
}
\item{
{\bf  Parameters}
  \begin{itemize}
   \item{
\texttt{event} -- the event to be parsed}
   \item{
\texttt{type} -- the type of the value to be fished out}
  \end{itemize}
}%end item
\item{{\bf  Returns} -- 
the value of the type type in the event 
}%end item
\end{itemize}
}%end item
\item{ 
\index{parseCalendar(String)}
{\bf  parseCalendar}\\
\texttt{public static CalEvent\lbrack \rbrack \ {\bf  parseCalendar}(\texttt{java.lang.String} {\bf  calendar})
\label{is.mpg.ruglan.iCalParser.parseCalendar(java.lang.String)}}%end signature
\begin{itemize}
\item{
{\bf  Description}

Takes a string containing an iCal calendar and returns a list of the events in the calendar
}
\item{
{\bf  Parameters}
  \begin{itemize}
   \item{
\texttt{calendar} -- A string of an iCal calendar}
  \end{itemize}
}%end item
\item{{\bf  Returns} -- 
a list of the events in the calendar cal 
}%end item
\end{itemize}
}%end item
\item{ 
\index{removeCR(String\lbrack \rbrack )}
{\bf  removeCR}\\
\texttt{public static java.lang.String\lbrack \rbrack \ {\bf  removeCR}(\texttt{java.lang.String\lbrack \rbrack } {\bf  lines})
\label{is.mpg.ruglan.iCalParser.removeCR(java.lang.String[])}}%end signature
\begin{itemize}
\item{
{\bf  Description}

Removes carriage returns in all string in an array of strings
}
\item{
{\bf  Parameters}
  \begin{itemize}
   \item{
\texttt{lines} -- the lines for which the carriage return is to be removed from}
  \end{itemize}
}%end item
\item{{\bf  Returns} -- 
 
}%end item
\end{itemize}
}%end item
\item{ 
\index{urlToCalEvents(String)}
{\bf  urlToCalEvents}\\
\texttt{public static CalEvent\lbrack \rbrack \ {\bf  urlToCalEvents}(\texttt{java.lang.String} {\bf  url})
\label{is.mpg.ruglan.iCalParser.urlToCalEvents(java.lang.String)}}%end signature
\begin{itemize}
\item{
{\bf  Parameters}
  \begin{itemize}
   \item{
\texttt{url} -- The url to the iCal calendar}
  \end{itemize}
}%end item
\item{{\bf  Returns} -- 
a list of the events in the calendar located at the url 
}%end item
\end{itemize}
}%end item
\item{ 
\index{urlToString(String)}
{\bf  urlToString}\\
\texttt{public static java.lang.String\ {\bf  urlToString}(\texttt{java.lang.String} {\bf  url})
\label{is.mpg.ruglan.iCalParser.urlToString(java.lang.String)}}%end signature
\begin{itemize}
\item{
{\bf  Description}

Takes an url and fetches its contents into a string
}
\item{
{\bf  Parameters}
  \begin{itemize}
   \item{
\texttt{url} -- the url of the website to be downloaded}
  \end{itemize}
}%end item
\item{{\bf  Returns} -- 
the string with the content of the url 
}%end item
\end{itemize}
}%end item
\end{itemize}
}
}
\subsection{\label{is.mpg.ruglan.rDataBase}Class rDataBase}{
\vskip .1in 
An helper class for an sqlite datebase All credit to Gaddo F. Benedetti\vskip .1in 
\subsubsection{Declaration}{
\small public class rDataBase
\\ {\bf  extends} SQLiteOpenHelper
\refdefined{.SQLiteOpenHelper}}
\subsubsection{Field summary}{
\begin{verse}
{\bf context} \\
{\bf DB\_NAME} \\
{\bf DB\_VERSION} \\
\end{verse}
}
\subsubsection{Constructor summary}{
\begin{verse}
{\bf rDataBase(Context)} \\
\end{verse}
}
\subsubsection{Method summary}{
\begin{verse}
{\bf onCreate(SQLiteDatabase)} \\
{\bf onUpgrade(SQLiteDatabase, int, int)} \\
\end{verse}
}
\subsubsection{Fields}{
\begin{itemize}
\item{
\index{DB\_VERSION}
\label{is.mpg.ruglan.rDataBase.DB_VERSION}static final int {\bf  DB\_VERSION}}
\item{
\index{DB\_NAME}
\label{is.mpg.ruglan.rDataBase.DB_NAME}static final java.lang.String {\bf  DB\_NAME}}
\item{
\index{context}
\label{is.mpg.ruglan.rDataBase.context} Context {\bf  context}}
\end{itemize}
}
\subsubsection{Constructors}{
\vskip -2em
\begin{itemize}
\item{ 
\index{rDataBase(Context)}
{\bf  rDataBase}\\
\texttt{public\ {\bf  rDataBase}(\texttt{Context} {\bf  context})
\label{is.mpg.ruglan.rDataBase(Context)}}%end signature
}%end item
\end{itemize}
}
\subsubsection{Methods}{
\vskip -2em
\begin{itemize}
\item{ 
\index{onCreate(SQLiteDatabase)}
{\bf  onCreate}\\
\texttt{public void\ {\bf  onCreate}(\texttt{SQLiteDatabase} {\bf  database})
\label{is.mpg.ruglan.rDataBase.onCreate(SQLiteDatabase)}}%end signature
}%end item
\item{ 
\index{onUpgrade(SQLiteDatabase, int, int)}
{\bf  onUpgrade}\\
\texttt{public void\ {\bf  onUpgrade}(\texttt{SQLiteDatabase} {\bf  db},
\texttt{int} {\bf  oldVersion},
\texttt{int} {\bf  newVersion})
\label{is.mpg.ruglan.rDataBase.onUpgrade(SQLiteDatabase, int, int)}}%end signature
}%end item
\end{itemize}
}
}
\subsection{\label{is.mpg.ruglan.SettingsActivity}Class SettingsActivity}{
\vskip .1in 
\subsubsection{Declaration}{
\small public class SettingsActivity
\\ {\bf  extends} Activity
\refdefined{.Activity}}
\subsubsection{Constructor summary}{
\begin{verse}
{\bf SettingsActivity()} \\
\end{verse}
}
\subsubsection{Method summary}{
\begin{verse}
{\bf onCreate(Bundle)} \\
{\bf onCreateOptionsMenu(Menu)} \\
{\bf onOptionsItemSelected(MenuItem)} \\
\end{verse}
}
\subsubsection{Constructors}{
\vskip -2em
\begin{itemize}
\item{ 
\index{SettingsActivity()}
{\bf  SettingsActivity}\\
\texttt{public\ {\bf  SettingsActivity}()
\label{is.mpg.ruglan.SettingsActivity()}}%end signature
}%end item
\end{itemize}
}
\subsubsection{Methods}{
\vskip -2em
\begin{itemize}
\item{ 
\index{onCreate(Bundle)}
{\bf  onCreate}\\
\texttt{protected void\ {\bf  onCreate}(\texttt{Bundle} {\bf  savedInstanceState})
\label{is.mpg.ruglan.SettingsActivity.onCreate(Bundle)}}%end signature
}%end item
\item{ 
\index{onCreateOptionsMenu(Menu)}
{\bf  onCreateOptionsMenu}\\
\texttt{public boolean\ {\bf  onCreateOptionsMenu}(\texttt{Menu} {\bf  menu})
\label{is.mpg.ruglan.SettingsActivity.onCreateOptionsMenu(Menu)}}%end signature
}%end item
\item{ 
\index{onOptionsItemSelected(MenuItem)}
{\bf  onOptionsItemSelected}\\
\texttt{public boolean\ {\bf  onOptionsItemSelected}(\texttt{MenuItem} {\bf  item})
\label{is.mpg.ruglan.SettingsActivity.onOptionsItemSelected(MenuItem)}}%end signature
}%end item
\end{itemize}
}
}
}
\end{document}
