\documentclass[11pt,a4paper]{report}
\usepackage{color}
\usepackage{ifthen}
\usepackage{ifpdf}
\usepackage[headings]{fullpage}
\ifpdf \usepackage[pdftex, pdfpagemode={UseOutlines},bookmarks,colorlinks,linkcolor={blue},plainpages=false,pdfpagelabels,citecolor={red},breaklinks=true]{hyperref}
  \usepackage[pdftex]{graphicx}
  \pdfcompresslevel=9
  \DeclareGraphicsRule{*}{mps}{*}{}
\else
  \usepackage[dvips]{graphicx}
\fi

\newcommand{\entityintro}[3]{%
  \hbox to \hsize{%
    \vbox{%
      \hbox to .2in{}%
    }%
    {\bf  #1}%
    \dotfill\pageref{#2}%
  }
  \makebox[\hsize]{%
    \parbox{.4in}{}%
    \parbox[l]{5in}{%
      \vspace{1mm}%
      #3%
      \vspace{1mm}%
    }%
  }%
}
\newcommand{\refdefined}[1]{
\expandafter\ifx\csname r@#1\endcsname\relax
\relax\else
{$($in \ref{#1}, page \pageref{#1}$)$}\fi}
\date{\today}
\title{rUglan documentation}
\author{Jon Arnar Tomasson, Matthias Pall Gissurarson and Sigurdur Fannar Vilhelmsson}
\chardef\textbackslash=`\\
\begin{document}
\maketitle
\sloppy
\addtocontents{toc}{\protect\markboth{Contents}{Contents}}
\tableofcontents
\chapter*{Class Hierarchy}{
\thispagestyle{empty}
\markboth{Class Hierarchy}{Class Hierarchy}
\addcontentsline{toc}{chapter}{Class Hierarchy}
\section*{Classes}
{\raggedright
\hspace{0.0cm} $\bullet$ java.lang.Object {\tiny \refdefined{java.lang.Object}} \\
\hspace{1.0cm} $\bullet$ is.mpg.ruglan.utils.Utils {\tiny \refdefined{is.mpg.ruglan.utils.Utils}} \\
}
}
\chapter{Package is.mpg.ruglan.utils}{
\label{is.mpg.ruglan.utils}\hskip -.05in
\hbox to \hsize{\textit{ Package Contents\hfil Page}}
\vskip .13in
\hbox{{\bf  Classes}}
\entityintro{Utils}{is.mpg.ruglan.utils.Utils}{Created by tritlo on 10/28/13.}
\vskip .1in
\vskip .1in
\section{\label{is.mpg.ruglan.utils.Utils}Class Utils}{
\vskip .1in 
Created by tritlo on 10/28/13.\vskip .1in 
\subsection{Declaration}{
\small public class Utils
\\ {\bf  extends} java.lang.Object
\refdefined{java.lang.Object}}
\subsection{Field summary}{
\begin{verse}
{\bf colors} \\
{\bf googleMapsLink} \\
{\bf hiddenColor} \\
{\bf iCalURLKey} \\
{\bf lastUpdateKey} \\
{\bf showHiddenDefaultValue} \\
{\bf showHiddenKey} \\
\end{verse}
}
\subsection{Constructor summary}{
\begin{verse}
{\bf Utils()} \\
\end{verse}
}
\subsection{Method summary}{
\begin{verse}
{\bf dateToCalendar(Date)} \\
{\bf displayMessage(String, String, Context)} \\
{\bf fillGoogleMapsLinkMap()} Initializes the googleMapsLink map.\\
{\bf getJavascriptForCalEvents(CalEvent\lbrack \rbrack , Context)} \\
{\bf getTextFromAssetsTextFile(String, Context)} \\
{\bf isLecture(String)} \\
{\bf isSameDay(Calendar, Calendar)} \\
{\bf setCalendarViewByOrientation(Context, WebView)} \\
{\bf stripCourseNumberFromName(String)} Removes the course number from of the course name\\
\end{verse}
}
\subsection{Fields}{
\begin{itemize}
\item{
\index{googleMapsLink}
\label{is.mpg.ruglan.utils.Utils.googleMapsLink}public static java.util.HashMap {\bf  googleMapsLink}}
\item{
\index{lastUpdateKey}
\label{is.mpg.ruglan.utils.Utils.lastUpdateKey}public static java.lang.String {\bf  lastUpdateKey}}
\item{
\index{iCalURLKey}
\label{is.mpg.ruglan.utils.Utils.iCalURLKey}public static java.lang.String {\bf  iCalURLKey}}
\item{
\index{showHiddenKey}
\label{is.mpg.ruglan.utils.Utils.showHiddenKey}public static java.lang.String {\bf  showHiddenKey}}
\item{
\index{showHiddenDefaultValue}
\label{is.mpg.ruglan.utils.Utils.showHiddenDefaultValue}public static boolean {\bf  showHiddenDefaultValue}}
\item{
\index{hiddenColor}
\label{is.mpg.ruglan.utils.Utils.hiddenColor}public static java.lang.String {\bf  hiddenColor}}
\item{
\index{colors}
\label{is.mpg.ruglan.utils.Utils.colors}public static java.lang.String {\bf  colors}}
\end{itemize}
}
\subsection{Constructors}{
\vskip -2em
\begin{itemize}
\item{ 
\index{Utils()}
{\bf  Utils}\\
\texttt{public\ {\bf  Utils}()
\label{is.mpg.ruglan.utils.Utils()}}%end signature
}%end item
\end{itemize}
}
\subsection{Methods}{
\vskip -2em
\begin{itemize}
\item{ 
\index{dateToCalendar(Date)}
{\bf  dateToCalendar}\\
\texttt{public static java.util.Calendar\ {\bf  dateToCalendar}(\texttt{java.util.Date} {\bf  date})
\label{is.mpg.ruglan.utils.Utils.dateToCalendar(java.util.Date)}}%end signature
\begin{itemize}
\item{
{\bf  Parameters}
  \begin{itemize}
   \item{
\texttt{date} -- Date object.}
  \end{itemize}
}%end item
\item{{\bf  Returns} -- 
A calendar object with the same date as the date parameter. 
}%end item
\end{itemize}
}%end item
\item{ 
\index{displayMessage(String, String, Context)}
{\bf  displayMessage}\\
\texttt{public static void\ {\bf  displayMessage}(\texttt{java.lang.String} {\bf  messageHeader},
\texttt{java.lang.String} {\bf  messageBody},
\texttt{Context} {\bf  ctx})
\label{is.mpg.ruglan.utils.Utils.displayMessage(java.lang.String, java.lang.String, Context)}}%end signature
}%end item
\item{ 
\index{fillGoogleMapsLinkMap()}
{\bf  fillGoogleMapsLinkMap}\\
\texttt{public static void\ {\bf  fillGoogleMapsLinkMap}()
\label{is.mpg.ruglan.utils.Utils.fillGoogleMapsLinkMap()}}%end signature
\begin{itemize}
\item{
{\bf  Description}

Initializes the googleMapsLink map.
}
\end{itemize}
}%end item
\item{ 
\index{getJavascriptForCalEvents(CalEvent\lbrack \rbrack , Context)}
{\bf  getJavascriptForCalEvents}\\
\texttt{public static java.lang.String\ {\bf  getJavascriptForCalEvents}(\texttt{is.mpg.ruglan.data.CalEvent\lbrack \rbrack } {\bf  events},
\texttt{Context} {\bf  context})
\label{is.mpg.ruglan.utils.Utils.getJavascriptForCalEvents(is.mpg.ruglan.data.CalEvent[], Context)}}%end signature
}%end item
\item{ 
\index{getTextFromAssetsTextFile(String, Context)}
{\bf  getTextFromAssetsTextFile}\\
\texttt{public static java.lang.String\ {\bf  getTextFromAssetsTextFile}(\texttt{java.lang.String} {\bf  filename},
\texttt{Context} {\bf  context})
\label{is.mpg.ruglan.utils.Utils.getTextFromAssetsTextFile(java.lang.String, Context)}}%end signature
}%end item
\item{ 
\index{isLecture(String)}
{\bf  isLecture}\\
\texttt{public static java.lang.Boolean\ {\bf  isLecture}(\texttt{java.lang.String} {\bf  description})
\label{is.mpg.ruglan.utils.Utils.isLecture(java.lang.String)}}%end signature
\begin{itemize}
\item{
{\bf  Parameters}
  \begin{itemize}
   \item{
\texttt{description} -- }
  \end{itemize}
}%end item
\item{{\bf  Returns} -- 
whether the description matches the description of a lecture. 
}%end item
\end{itemize}
}%end item
\item{ 
\index{isSameDay(Calendar, Calendar)}
{\bf  isSameDay}\\
\texttt{public static java.lang.Boolean\ {\bf  isSameDay}(\texttt{java.util.Calendar} {\bf  cal1},
\texttt{java.util.Calendar} {\bf  cal2})
\label{is.mpg.ruglan.utils.Utils.isSameDay(java.util.Calendar, java.util.Calendar)}}%end signature
\begin{itemize}
\item{
{\bf  Parameters}
  \begin{itemize}
   \item{
\texttt{cal1} -- }
   \item{
\texttt{cal2} -- }
  \end{itemize}
}%end item
\item{{\bf  Returns} -- 
True IFF cal1 and cal2 are the same day. 
}%end item
\end{itemize}
}%end item
\item{ 
\index{setCalendarViewByOrientation(Context, WebView)}
{\bf  setCalendarViewByOrientation}\\
\texttt{public static void\ {\bf  setCalendarViewByOrientation}(\texttt{Context} {\bf  context},
\texttt{WebView} {\bf  wv})
\label{is.mpg.ruglan.utils.Utils.setCalendarViewByOrientation(Context, WebView)}}%end signature
\begin{itemize}
\item{
{\bf  Parameters}
  \begin{itemize}
   \item{
\texttt{c} -- The Current context of the application.}
   \item{
\texttt{wv} -- The WebView that contains the FullCalendar object. Sets the agenda view as Week if the orientation is landscape and to Day if the orientation is portrait.}
  \end{itemize}
}%end item
\end{itemize}
}%end item
\item{ 
\index{stripCourseNumberFromName(String)}
{\bf  stripCourseNumberFromName}\\
\texttt{public static java.lang.String\ {\bf  stripCourseNumberFromName}(\texttt{java.lang.String} {\bf  courseName})
\label{is.mpg.ruglan.utils.Utils.stripCourseNumberFromName(java.lang.String)}}%end signature
\begin{itemize}
\item{
{\bf  Description}

Removes the course number from of the course name
}
\item{
{\bf  Parameters}
  \begin{itemize}
   \item{
\texttt{courseName} -- to remove course number from}
  \end{itemize}
}%end item
\item{{\bf  Returns} -- 
the course name but with the course name removed. 
}%end item
\end{itemize}
}%end item
\end{itemize}
}
}
}
\end{document}
